\documentclass{article}

\usepackage{amsmath}
\usepackage{amssymb}
\usepackage{amsthm}

\usepackage[pdftex]{graphicx}
\usepackage[english]{babel}
\usepackage{hyperref}
\usepackage{cleveref}
\usepackage{url}

\selectlanguage{english}
\allowdisplaybreaks[1]

\newtheorem{theorem}{Theorem}[section]
\newtheorem{lemma}[theorem]{Lemma}

\theoremstyle{definition}
\newtheorem{definition}[theorem]{Definition}
\newtheorem{example}[theorem]{Example}
\newtheorem{xca}[theorem]{Exercise}

\theoremstyle{remark}
\newtheorem{remark}[theorem]{Remark}

%Mathematical shorthand.
\newcommand{\N}{\mathbf{N}}
\newcommand{\Z}{\mathbf{Z}}
\newcommand{\R}{\mathbf{R}}
\newcommand{\C}{\mathcal{C}}
\newcommand{\setsep}{\ |\ }
\newcommand{\Ort}{\mathrm{O}}
\newcommand{\SO}{\mathrm{SO}}
\newcommand{\vecspan}{\mathrm{span}}
\newcommand{\ud}{\, \mathrm{d}}
\newcommand{\smooth}{\mathrm{C}^{\infty}}
\newcommand{\Div}{\mathrm{div} \,}
\newcommand{\Rot}{\mathrm{rot} \,}
\newcommand{\Grad}{\mathrm{grad} \,}
\newcommand{\Laplace}{\Delta}
\newcommand{\Alambert}{\box}
\newcommand{\implystep}[2]{\stackrel{\labelcref{#1}}{#2}}

\begin{document}

\title{Manual of the \texttt{tiny-game-engine}}

\author{B.~O.~Fagginger Auer}

\date{\today}

\maketitle

\section{Shading}

TODO: Add image.
Consider a single point $p \in \R^3$ on the surface of an object that we want to illuminate.
Let $n \in S^2$ be the normal of the surface at $p$.
Let $l \in S^2$ be the normalised direction from $p$ to the light source illuminating the object.
Let $v \in \R^3$ be the position of the viewer and define the direction from $p$ to the viewer as
$$e := (v - p)/\|v - p\|.$$
Let $l_\perp$ and $e_\perp$ be the normalised projections of $l$ and $e$ to the tangent space of the object at $p$, i.e., the normalisation of $l - \langle n, l \rangle n$ and $e - \langle n, e \rangle n$, respectively.
Let $\alpha$ be the angle between $n$ and $l$, $\beta$ the angle between $n$ and $e$, and $\gamma$ the angle between $l_\perp$ and $e_\perp$.

Since we can see $p$ from $v$, we will assume that $0 \leq \beta \leq \frac{\pi}{2}$ (otherwise $p$ is invisible due to backface culling).

\subsection{Diffuse shading}

The most simple shading model for diffuse reflections is the cosine emission law due to Lambert.
Here, the intensity $I_d$ of the diffuse colour of an object is given by
\begin{equation} \label{eq:shading:lambert}
I_d = \max(0, \cos(\alpha)) = \max(0, \langle n, l \rangle).
\end{equation}
However, this model is too simple to properly calculate reflections from rough diffuse objects (such as the moon).

To improve upon the Lambertian reflection model, we will implement the qualitative Oren--Nayar reflectance model from \cite[\S 4.4]{Oren1994}.
Here, we have
\begin{equation} \label{eq:shading:orennayar}
I_d = \max(0, \cos(\alpha)) \, \left( A + B \, \max(0, \cos(\gamma)) \, \sin(\delta_1) \, \tan(\delta_2) \right),
\end{equation}
where
\begin{align*}
A & = 1 - 0.5 \, \frac{\sigma^2}{\sigma^2 + 0.57}, \\
B & = 0.45 \, \frac{\sigma^2}{\sigma^2 + 0.09}, \\
\delta_1 & = \max(\alpha, \beta), \\
\delta_2 & = \min(\alpha, \beta),
\end{align*}
and $\sigma \in \R_{\geq 0}$ is the surface roughness.

Let us simplify this expression.
Note that $0 \leq \alpha, \beta \leq \frac{\pi}{2}$, because otherwise there is either no diffuse contribution, or the point $p$ is invisible due to backface culling.
Therefore $\cos(\alpha) \leq \cos(\beta)$ if and only if $\alpha \geq \beta$, so $\cos(\delta_2) = \cos(\min(\alpha, \beta)) = \max(\cos(\alpha), \cos(\beta))$.
As $\sin(\arccos(x)) = \sqrt{1 - x^2}$, we have that $\sin(\alpha) = \sin(\arccos(\langle n, l \rangle)) = \sqrt{1 - \langle n, l \rangle^2}$, and similarly for $\sin(\beta)$.
Hence,
\begin{align*}
\sin(\delta_1) \, \tan(\delta_2)
 & = \sin(\delta_1) \, \sin(\delta_2) / \cos(\delta_2) \\
 & = \sin(\alpha) \, \sin(\beta) / \cos(\min(\alpha, \beta)) \\
 & = \sin(\alpha) \, \sin(\beta) / \max(\cos(\alpha), \cos(\beta)) \\
 & = \sqrt{(1 - \langle n, l \rangle^2) \, (1 - \langle n, e \rangle^2)} / \max(\langle n, l \rangle, \langle n, l \rangle).
\end{align*}
Therefore, we can calculate $I_d$ as
\begin{equation}
I_d = \max(0, \langle n, l \rangle) \, \left( A + B \, \max(0, \langle l_\perp, e_\perp \rangle) \, \frac{\sqrt{(1 - \langle n, l \rangle^2) \, (1 - \langle n, e \rangle^2)}}{\max(\langle n, l \rangle, \langle n, l \rangle)} \right).
\end{equation}

\bibliographystyle{plain}
\bibliography{manual}

\end{document}

